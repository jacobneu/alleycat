% Jacob Neumann

% DOCUMENT CLASS AND PACKAGE USE
    \documentclass[11pt]{book}
    %\usepackage{paper}

    \author{Jacob Neumann}
    \title{Alley Cats and Boxes}

\begin{document}

\maketitle

\tableofcontents


\section{Introduction}

\section{About This Book}


\chapter{Classical Theory of Relations}

\section{Expressions, Sets, Functions, and Relations}

\section{Partially-Ordered Sets}

\section{Comparing and Composing Relations}

\section{Transposition and Inverse}

\section{Functions versus Relations}

\section{Equivalence Relations and Quotients}


\chapter{Modal Logic}

\addcontentsline{toc}{section}{Review: Classical Logic}
\section*{Review: Classical Logic}

\section{Motivation}

\section{Kripke Semantics for Basic Modal Logic}

\section{Bisimulations}

\section{Propositional Dynamic Logic}

\section{Modal Logic of Partial Functions}

\section{Topological Modal Logic}

\section{Kripke Frames and Definability}

\section{First-Order Modal Logic}

\section{Kripke Semantics for Intuitionistic Logic}


\chapter{Allegories}

\addcontentsline{toc}{section}{Review: Category Theory}
\section*{Review: Category Theory}

\section{Modular Lattices}

\section{Allegories}

\section{Back-and-Forth Classes}

\section{Regular Categories and Allegories}

\section{Tabulation, Division, and Distribution}

\section{Power and Quotient Allegories}

\section{Syntactic Allegories}

\section{Enrichment}

\section{Bicategories of Relations}

\section{Monoidal Categories}

\chapter{Advanced Topics}

\addcontentsline{toc}{section}{Review: Dependent Type Theory}
\section*{Review: Dependent Type Theory}

\section{The Allegory of Categories}

\section{Allegories in Homotopy Type Theory}

\section{Dynamic Topological Logic}

\section{Program Construction and Refined Frame Theory}

\section{Topological First-Order Modal Logic}

\section{IPL and STLC}

\section{Regular Logic}

\section{Kripke-Joyal Semantics}

\section{Hyperdoctrines}

\section{Reflective Subcategories and Modalities}

\section{(Co)monads and Modality}

\section{Modal Type Theory}

\end{document}

